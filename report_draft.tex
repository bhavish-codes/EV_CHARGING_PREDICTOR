\documentclass[12pt]{article}
\usepackage[utf8]{inputenc}
\usepackage{amsmath}
\usepackage{graphicx}
\usepackage{hyperref}
\usepackage{geometry}
\geometry{a4paper, margin=1in}

\title{Predictive Analytics for Urban EV Charging Infrastructure}
\author{Project Lead}
\date{February 2026}

\begin{document}

\maketitle

\begin{abstract}
This report details the design and implementation of an AI-driven system for predicting electric vehicle (EV) charging demand. Leveraging the UrbanEV benchmark dataset, we develop a Random Forest regression model for demand forecasting. The final system provides a Streamlit-based interface for visualizing load patterns and exploring demand forecasts.
\end{abstract}

\section{Introduction}
As the global adoption of electric vehicles accelerates, urban charging infrastructure faces significant load management challenges. This project focuses on historical data analysis and demand forecasting to assist in infrastructure optimization.

\section{Methodology}

\subsection{Data Preprocessing}
The system utilizes the UrbanEV dataset, featuring 5-minute interval charging sessions from stations in Shenzhen, China. Data was aggregated to hourly intervals, and features such as hour-of-day, day-of-week, and pricing metrics were engineered.

\subsection{Demand Prediction Model}
A Random Forest Regressor was selected for its robustness in handling non-linear time-series features. 
\begin{equation}
\hat{y} = \frac{1}{N} \sum_{i=1}^{B} T_i(x)
\end{equation}
The model achieved a Mean Absolute Error (MAE) of 7.35 kWh.

\section{Results}
The Streamlit dashboard allows users to visualize station locations and predict future demand cycles based on user-defined parameters such as time and pricing.

\section{Conclusion}
The proposed system demonstrates the efficacy of using Random Forest models for charging demand optimization. Future work involves integrating geographic expansion recommendations and real-time load balancing.

\end{document}
